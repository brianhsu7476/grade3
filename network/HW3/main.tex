%\documentclass[12pt, a4paper,oneside]{book}
\documentclass[12pt, a4paper]{article}

\usepackage[margin=3cm]{geometry}
\usepackage{amsmath,amsthm,amssymb,mathrsfs}
%\usepackage{yhmath}
\usepackage{graphicx}
\usepackage{fontspec}
\usepackage{type1cm}
\usepackage{titlesec}
\usepackage{titling}
\usepackage{fancyhdr}
\usepackage{tabularx}
\usepackage[square, comma, numbers, super, sort&compress]{natbib}
\usepackage[unicode=true, pdfborder={0 0 0}, bookmarksdepth=-1]{hyperref}
\usepackage[noabbrev,capitalize]{cleveref}
\usepackage[usenames, dvipsnames]{color}
\usepackage{enumerate}
\usepackage{tikz-cd}
\usepackage{centernot}
\usepackage{cases}
\usepackage{indentfirst}
\usepackage{tkz-berge}
\usepackage{tkz-graph}
\usetikzlibrary{arrows, petri, topaths}

%\usepackage{minted, listings}
\usepackage{listings}
\lstset{
    basicstyle=\ttfamily,
    breaklines,
    backgroundcolor = \color{lightgray},
    numbers=left
}
%\newminted[cpp]{c++}{
    %frame=single,
    %frame=lines,
    %framesep=0mm,
    %breaklines,
    %linenos,
    %mathescape,
    %xleftmargin=1cm,
    %baselinestretch=1.6,
    %style=manni,
    %fontsize=,
%}

\usepackage{xeCJK}
    \setCJKmainfont[AutoFakeBold=6,AutoFakeSlant=.2]{AR PL KaitiM Big5}
    \XeTeXlinebreaklocale "zh"
    \XeTeXlinebreakskip = 0pt plus 1pt

\usepackage{tocloft}
    \renewcommand{\cftsecleader}{\cftdotfill{\cftdotsep}}
    \setcounter{tocdepth}{1}

\renewcommand{\baselinestretch}{1.5}

\setlength{\headheight}{15pt}
\setlength{\droptitle}{-1.5cm}

\renewcommand{\contentsname}{目錄}
\renewcommand{\bibsection}{\bf\Large 參考資料\normalsize\rm}
\renewcommand{\bibname}{\bf 參考資料}

\hyphenation{word list}

\newcommand{\A}{\mathcal{A}}
\newcommand{\B}{\mathcal{B}}
\renewcommand{\b}{\overline}
\newcommand{\C}{\mathbb{C}}
\newcommand{\D}{\mathrm{D}}
\newcommand{\E}{\mathrm{E}}
\renewcommand{\H}{\mathrm{H}}
\newcommand{\I}{\mathbb{I}}
\newcommand{\N}{\mathbb{N}}
\renewcommand{\P}{\mathrm{Pr}}
\newcommand{\Q}{\mathbb{Q}}
\newcommand{\R}{\mathbb{R}}
\newcommand{\V}{\Vertex}
\newcommand{\Z}{\mathbb{Z}}
\newcommand{\simple}{\SetVertexSimple[MinSize=8pt]}
\newcommand{\pictures}{\includegraphics[height=6cm]}
\newcommand{\directed}{\tikzstyle{EdgeStyle}=[post]}
\newcommand{\bidirected}{\tikzstyle{EdgeStyle}=[pre and post]}
\newcommand{\degree}{^\circ}
\newcommand{\Arc}[1]{\wideparen{{#1}}}
\newcommand{\lin}{\overleftrightarrow}
\newcommand{\ray}{\overrightarrow}
\newcommand{\seg}{\overline}
\newcommand{\ang}{\angle}
\newcommand{\inv}{^{-1}}
\newcommand{\dang}{\measuredangle}
\newcommand{\then}{\Rightarrow}
\newcommand{\cuz}{\because}
\newcommand{\so}{\therefore}
%\newcommand{\D}[3][2]{\left|\begin{array}{cc}#2&#3\\\b{#2}&\b{#3}\end{array}\right|}
\newcommand{\spc}{\;\;\;\;\;\;}
\newcommand{\st}{\text{ s.t. }}
\newcommand{\claim}{\textit{Claim}}
\renewcommand{\ll}{\longleftarrow}
\newcommand{\lr}{\longrightarrow}
\renewcommand{\mod}{\mathrm{\ mod\ }}
\newcommand{\dep}{\mathrm{depth}}
\newcommand{\contradict}{\rightarrow\leftarrow}
\newcommand{\deter}[9]{\left|\begin{array}{ccc}#1&#2&#3\\#4&#5&#6\\#7&#8&#9\end{array}\right|}
\newcommand{\psim}{\stackrel{+}{\sim}}
\newcommand{\msim}{\stackrel{-}{\sim}}
\newcommand{\ma}{\measuredangle}
\newcommand{\parato}{\,\|\,}
\DeclareMathOperator{\lcm}{lcm}
\DeclareMathOperator{\Spec}{Spec}
\newcommand{\erf}{\mathrm{erf}}
\newcommand{\Ker}{\mathrm{Ker}}
\renewcommand{\Im}{\mathrm{Im}}
\newcommand{\Mn}{\mathrm{Mn}}
\renewcommand{\dim}{\mathrm{dim}}
\newcommand{\rank}{\mathrm{rank}}
\renewcommand{\matrix}[1]{\begin{pmatrix}#1\end{pmatrix}}
\newcommand{\nmatrix}[1]{\begin{array}{ccccccccccccccccccccccccccc}#1\end{array}}
\newcommand{\nullity}{\mathrm{nullity}}
\newcommand{\adj}{\mathrm{adj}}
\newcommand{\Id}{\mathrm{Id}}
\newcommand{\Tr}{\mathrm{Tr}}
\newcommand{\ch}{\mathrm{ch}}
\newcommand{\Span}{\mathrm{span}}
\newcommand{\inn}[2]{\left\langle#1,#2\right\rangle}
\newcommand{\abs}[1]{\left\|#1\right\|}
\newcommand{\rfeq}[1]{\textbf{Equation (\ref{#1})}}
\newcommand{\rflm}[1]{\textbf{Lemma (\ref{#1})}}
\newcommand{\overtext}[1]{\overset{\text{#1}}}
\newcommand{\Proj}{\mathrm{Proj}}
\newcommand{\suml}{\sum\limits}
\newcommand{\prodl}{\prod\limits}
\newcommand{\capl}{\bigcap\limits}
\newcommand{\cupl}{\bigcup\limits}
\newcommand{\Null}{\mathrm{Null}}
\renewcommand{\sp}[1]{\abs{#1}_{\mathrm{sp}}}
\newcommand{\diag}{\mathrm{diag}}


\newtheoremstyle{mystyle}
  {15pt}{15pt}
  {}
  {}
  {\bf}
  {.}
  {1em}
  {}

\theoremstyle{mystyle}
\newtheorem{thm}{Theorem}[]
\newtheorem{thr}[thm]{引理}%alias
\newtheorem{ex}[thm]{Example}
\newtheorem{pr}{Problem}[]
\newtheorem{sbpr}{Subproblem}[pr]
\newtheorem{df}[thm]{Definition}
\newtheorem{pp}[thm]{Proposition}
\newtheorem{co}[thm]{Corollary}
\newtheorem{for}[thm]{Formula}
\newtheorem{lm}{Lemma}[pr]
\newtheorem{nota}[thm]{Notation}
\newtheorem{cl}{Claim}[pr]
\newtheorem{rmk}{Remark}[]
\numberwithin{equation}{pr}

\newtheoremstyle{pfstyle}
  {0pt}{3pt}
  {}
  {}
  {\it}
  {.}
  {1em}
  {}

\theoremstyle{pfstyle}
%\newtheorem{cl}{Claim}[pr]
\newtheorem*{pf}{pf}
\newtheorem*{tht}{想法}
%\newtheorem*{rmk}{\bf 註}

\newcommand{\real}{\mathbb{R}}
\newcommand{\rational}{\mathbb{Q}}
\newcommand{\complex}{\mathbb{C}}
\newcommand{\integer}{\mathbb{Z}}
\newcommand{\nat}{\mathbb{N}}

%\newenvironment{solution}
%    {\renewcommand\qedsymbol{$\blacksquare$}\begin{proof}[Solution]}
%    {\end{proof}}
%\newenvironment{sketch}
%    {\begin{proof}[Sketch of Proof]}
%    {\end{proof}}
%\ExplSyntaxOn
%\NewDocumentCommand{\cycle}{ O{\;} m }
% {
%  (
%  \alec_cycle:nn { #1 } { #2 }
%  )
% }

%\seq_new:N \l_alec_cycle_seq
%\cs_new_protected:Npn \alec_cycle:nn #1 #2
% {
%  \seq_set_split:Nnn \l_alec_cycle_seq { , } { #2 }
%  \seq_use:Nn \l_alec_cycle_seq { #1 }
% }
%\ExplSyntaxOff

\renewcommand\qedsymbol{$\blacksquare$}

\everymath{\displaystyle}

\begin{document}

    %\settowidth{\parindent}{一二}
%\setlength{\parindent}{24pt}
\setlength{\parindent}{0pt}

\pagenumbering{gobble}
\title{機率與統計 HW1}
\author{許博翔}
\date{\today}
\maketitle
%\tableofcontents
\thispagestyle{empty}
\pagenumbering{arabic}
\setcounter{page}{1}
\pagestyle{fancy}

\renewcommand{\sectionmark}[1]{\markright{#1}}
\renewcommand{\subsectionmark}[1]{}

\lhead{\thetitle}
\chead{}
\rhead{}
\lfoot{Author:~\theauthor}
\cfoot{}
\rfoot{\thepage}
\renewcommand{\headrulewidth}{0.4pt}
\renewcommand{\footrulewidth}{0.4pt}

%\setcounter{chapter}{1}
%\setcounter{section}{-1}

%\mainmatter
\begin{multicols}{3}
金融科技使得金融服務具有\\
  效率性 \\
  普及性 \\
正確!\\
  皆是 \\
 \\
問題 2\\
5 / 5 分\\
目前CFA 考慮的 Fintech 範圍不包含哪項?\\
  data analytics \\
  robo advisor \\
正確!\\
  crowd funding \\
 \\
問題 3\\
5 / 5 分\\
以下哪些屬於另類資料 (alternative data),與傳統金融的結構化資料不同?\\
  社群留言 \\
  衛星影像 \\
  大氣微粒子 \\
正確!\\
  以上皆是 \\
 \\
問題 4\\
5 / 5 分\\
另類資料的哪些特色使得數據分析較為困難?\\
  數據量大 \\
  複雜度高 \\
  資料清洗不易 \\
正確!\\
  以上皆是 \\
 \\
問題 5\\
5 / 5 分\\
「以人為本的可信任AI」是為了回應AI預測可能造成某些特定的風險的誤判風險,具體作法包括:\\
  謹慎開發或使用AI這類強大的工具以避免偏誤 \\
  將風險就使用場合、可能造成的損害,以及損害發生的機率等做分級 \\
  聚焦高度風險的監理,例如信用風險等,以避免過度監管 \\
正確!\\
  以上皆是 \\
 \\
問題 6\\
0 / 5 分\\
平台經濟模式的目標是?\\
正確答案\\
  金融生態圈 \\
您已回答\\
  數位轉型 \\
  跨域人才培養 \\
  AI 技術進步 \\
 \\
問題 7\\
5 / 5 分\\
機器人理專屬於金融六大功能的哪一項?\\
正確!\\
  投資管理 \\
  籌資 \\
  支付 \\
 \\
問題 8\\
5 / 5 分\\
機器人理專主要交易何種標的資產?\\
  股票 \\
  基金 \\
正確!\\
  ETF \\
 \\
問題 9\\
5 / 5 分\\
以下那一項不是現行機器人理專缺點\\
  無面對面服務 \\
正確!\\
  低波動 \\
  缺乏客製化 \\
 \\
問題 10\\
0 / 5 分\\
機器人理專提供的資產配置是\\
您已回答\\
  靜態的 \\
正確答案\\
  動態的 \\
  高波動 \\
  費用率高 \\
 \\
問題 11\\
5 / 5 分\\
在每個風險下找出最佳的投資組合,這些投資組合預期風險與預期報酬的點集合可畫出的曲線是\\
  無異曲線 \\
  資本市場線 \\
正確!\\
  效率前緣 \\
  證券市場線 \\
 \\
問題 12\\
5 / 5 分\\
假設投資組合A和B有相同的平均報酬,相同的報酬標準差,但投資組合A相較於投資組合B有較低的貝塔值(beta)。以夏普比例(Sharpe ratio)而言,投資組合A的表現:\\
  優於投資組合B的表現 \\
正確!\\
  相同於投資組合B的表現 \\
  劣於投資組合B的表現 \\
  資訊不足以判斷此問題 \\
 \\
問題 13\\
5 / 5 分\\
ETF 具有以下哪些特質?\\
  股票 \\
  基金 \\
正確!\\
  皆有 \\
 \\
問題 14\\
5 / 5 分\\
目前世界最大 AUM 的 ETF 是?\\
正確!\\
  SPY \\
  VOO \\
  SSO \\
 \\
問題 15\\
0 / 5 分\\
槓反型ETF以操作哪種資產為主?\\
正確答案\\
  期貨 \\
您已回答\\
  股票 \\
  債券 \\
 \\
問題 16\\
5 / 5 分\\
ETF 的風險包括\\
  交易對手 \\
  稅務 \\
正確!\\
  皆是 \\
 \\
問題 17\\
5 / 5 分\\
衡量期貨ETF 績效的方式有\\
  效率性 (efficiency) \\
  交易性 (tradability) \\
  適配性 (fit) \\
正確!\\
  皆可 \\
 \\
問題 18\\
0 / 5 分\\
若是反向兩倍ETF所連結的指數第一天上漲10%,第二天下跌10%,此ETF累積之報酬率為?\\
正確答案\\
  -4% \\
  -2% \\
  0% \\
您已回答\\
  2% \\
 \\
問題 19\\
5 / 5 分\\
微保險的特色包括\\
  低保險給付 \\
  快速理賠 \\
正確!\\
  皆是 \\
 \\
問題 20\\
5 / 5 分\\
物聯網在保險上的應用流程不包括\\
  使用感測器 \\
  大數據分析 \\
  風險評估 \\
正確!\\
  理賠 \\
  \\
外匯遠期契約可以視為一種線性的商品\\
正確!\\
  正確 \\
  錯誤 \\
 \\
問題 2\\
0 / 5 分\\
理論上股指期貨價格應比現貨價格高\\
正確答案\\
  正確 \\
您已回答\\
  錯誤 \\
 \\
問題 3\\
5 / 5 分\\
金融衍生品只有歐式買權與賣權兩種形式\\
  正確 \\
正確!\\
  錯誤 \\
 \\
問題 4\\
5 / 5 分\\
選擇權市場與債券市場可以互相套利\\
正確!\\
  正確 \\
  錯誤 \\
 \\
問題 5\\
5 / 5 分\\
二元樹模型是離散時間下的選擇權定價模型\\
正確!\\
  正確 \\
  錯誤 \\
 \\
問題 6\\
5 / 5 分\\
何者是金融工程領域中所榮獲的諾貝爾獎\\
  投資組合理論 \\
  選權權定價 \\
正確!\\
  以上皆是 \\
 \\
問題 7\\
0 / 5 分\\
譽為數理金融之父的是\\
您已回答\\
  Black-Scholes \\
  Richard Feynman \\
正確答案\\
  Louis Bachelier \\
 \\
問題 8\\
0 / 5 分\\
金融工程的典範機構為\\
正確答案\\
  CBOE \\
  IMF \\
您已回答\\
  WEF \\
 \\
問題 9\\
5 / 5 分\\
期貨採取哪種交易方式以規避違約風險\\
  保證金制度 \\
  每日結算 \\
正確!\\
  以上皆是 \\
 \\
問題 10\\
5 / 5 分\\
期貨市場不具備哪一項功能\\
  投機 \\
正確!\\
  穩定金融市場 \\
  避險 \\
 \\
問題 11\\
5 / 5 分\\
利用哪種合約可以轉換公司間的資產或負債\\
正確!\\
  交換 (swap) \\
  遠期 (forward) \\
  期貨 (futures) \\
 \\
問題 12\\
5 / 5 分\\
選擇權的報酬函數主要呈現\\
  線性 \\
正確!\\
  非線性 \\
  皆可 \\
 \\
問題 13\\
5 / 5 分\\
哪種契約規範了買方有權利,但非義務,可以在某固定的到期日 (maturity, expiration date) ,以一個預先決定好了的履約價格 ,來買入標的資產\\
正確!\\
  買權 \\
  賣權 \\
  期貨 \\
 \\
問題 14\\
0 / 5 分\\
選擇權契約的賣方在取得了權利金之後,如何支付未來可能的報酬問題,屬於哪類\\
您已回答\\
  定價 \\
正確答案\\
  避險 \\
  套利 \\
 \\
問題 15\\
5 / 5 分\\
何者交易行為造成閃崩\\
  套利 \\
  配對 \\
正確!\\
  高頻 \\
 \\
問題 16\\
5 / 5 分\\
以下何種數學模型適合股價\\
  布朗運動(Brownian motion) \\
正確!\\
  幾何布朗運動 (Geometric Brownian motion) \\
  Poisson processes \\
 \\
問題 17\\
0 / 5 分\\
風險中立評價法主要是基於哪個假設\\
您已回答\\
  最佳複製 \\
正確答案\\
  完美複製 \\
  最小變異 \\
 \\
問題 18\\
5 / 5 分\\
選擇權定價必須在哪種機率測度(probability measure)之下\\
正確!\\
  風險中立機率測度 (Risk-neutral probability measure) \\
  歷史機率測度 (Historical probability measure) \\
  皆可 \\
 \\
問題 19\\
5 / 5 分\\
最類似於Black-Scholes pricing PDE是哪一種\\
正確!\\
  熱傳導方程式 (heat equation) \\
  歷史機率測度 (Historical probability measure) \\
  皆可 \\
 \\
問題 20\\
5 / 5 分\\
Black-Scholes option pricing theory 中哪個量最常用來衡量選擇權的風險\\
  履約價 (Strike price) \\
  到期日 (Expiration date) \\
正確!\\
  波動率 (Volatility) \\
 \\
問題 21\\
0 / 10 分\\
按單期二元定價模型,若期初股價為1元,一季內上漲與下跌幅度皆為10%,履約價亦為1元,無風險利率為0,一季後到期的歐式賣權的理論價格應為? (四捨五入至小數點後第二位 Ex. 答案若為 0.123,則填 0.12;若為 0.1,就填 0.10 即可)\\
 \\
您已回答\\
1.11\\
正確答案\\
0.05 \\
\end{multicols}

\begin{pr}[11.9.5]
Let $A, B$ be the space of the strategy of the row, column player, respectively.\\
Suppose that the $3$ Nash equilibria are $(a_1, b_1), (a_2, b_2), (a_3, b_3)$.\\
Let $p_i(s), q_i(t)$ denote the probability that the row, column player plays $s, t$ when using strategy $a_i, b_i$, respectively.\\
Let $f:T\to A, g:S\to B$ denote the strategy of the row, column player when playing the game for the second time, where $f_i, g_i$ denote the constant function $f_i(t)=a_i, g_i(s)=b_i$, respectively.\\
Let $u(a, b), v(a, b)$ denote the expected payoff of the row, column player when the row player uses strategy $a$ and the column player uses strategy $b$, respectively.\\
%Let $a_ia_j$ denote the strategy of the row player that plays $a_i$ in the first game and plays $a_j$ in the second game no matter what the column player plays in the first game.\\
%Let $b_ib_j$ denote the strategy of the column player that plays $b_i$ in the first game and plays $b_j$ in the second game no matter what the row player plays in the first game.\\
Claim: $\forall i, j,\ (a_if_j, b_ig_j)$ are Nash equilibria.\\
Proof: $\forall a\in A,\ f:T\to A,\ u(af, b_ig_j)\overset{\forall s,\ g_j(s)=b_j}=u(a, b_i)+\sum_{t\in T}q_i(t)u(f(t), b_j)\overset{(a_j, b_j)\text{ is a Nash equilibria}}\leq u(a, b_i)+\sum_{t\in T}q_i(t)u(a_j, b_j)\overset{\sum_{t\in T}q_i(t)=1}=u(a, b_i)+u(a_j, b_j)\overset{(a_i, b_i)\text{ is a Nash equilibria}}\leq u(a_i, b_i)+u(a_j, b_j)=u(a_if_j, b_ig_j)$.\\
Similarly, $\forall b\in B,\ g:S\to B,\ v(a_if_j, bg)\leq v(a_if_j, b_ig_j)$.\\
$\so$ by the definition, $(a_if_j, b_ig_j)$ is a Nash equilibria.

$\so$ the claim holds. Since there are $9$ ways to choose $i, j$, there are at least $9$ Nash equilibria.
\end{pr}

\begin{pr}$ $
\begin{enumerate}[(a)]
\item Let $X\sim P$.\\
$\D(P\|G(p))=\suml_{x=1}^\infty P(x)\log\frac{P(x)}{Q(x)}=\suml_{x=1}^\infty P(x)\log\frac{P(x)}{(1-p)p^{x-1}}=\H(X)-\E[\log((1-p)p^{X-1})]=\H(X)-\log(1-p)-\E[(X-1)\log(p)]=H(X)-\log(1-p)-\log(p)E[X-1]=H(X)-\log(1-p)+\log p-\mu\log p$.\\
$\frac d{dp}\D(P\|G(p))=\frac1{1-p}+\frac1p-\frac1p\mu=\frac{1-(1-p)\mu}{p(1-p)}$, which equals to $0\iff\frac1{1-p}=\mu\iff p=1-\frac1\mu$.\\
One can also verify that if $p<1-\frac1\mu$, $\frac d{dp}\D(P\|G(p))<0$ and if $p>1-\frac1\mu$, $\frac d{dp}\D(P\|G(p))>0$.\\
$\so$ the minimum possible value of $\D(P\|G(p))$ occurs when $p=1-\frac1\mu$, that is, the distribution is $G(1-\frac1\mu)$, and $\D(P\|G(p))=H(X)-\log\mu+(1-\mu)\log(1-\mu)$.
\item Let $X_i\sim P_i, Y\sim R$ where $R(y):=\frac1m\suml_{i=1}^mP_i(y)$.\\
From HW2 we know that $H(R)\leq-\suml_{j=1}^\infty R(j)\log Q(j)$, with equality $\iff Q\sim R$.\\
$\then\suml_{i=1}^m\D(P_i\|Q)=\suml_{i=1}^m\left(H(X_i)-\suml_{j=1}^\infty P_i(j)\log Q(j)\right)=\suml_{i=1}^mH(X_i)-\suml_{j=1}^\infty\left(\suml_{i=1}^m P_i(j)\right)\log Q(j)=\suml_{i=1}^mH(X_i)-m\suml_{j=1}^\infty R(j)\log Q(j)\geq\suml_{i=1}^mH(X_i)-mH(R)$.\\
$\so\min_{Q\in\mathcal{P}(X)}\suml_{i=1}^mD(P_i\|Q)=\suml_{i=1}^mH(X_i)-mH(R)$, with minimizer $Q=R$, that is, $Q(y)=\frac 1m\suml_{i=1}^mP_i(y)$.
\end{enumerate}
\end{pr}

\begin{pr}$ $
\renewcommand{\Q}{\mathbf{Q}}
\newcommand{\q}{\mathbf{q}}
In HW2, we know that if $\sum_ip_i=\sum_iq_i=1$ where $p_i, q_i\geq0$, then $\sum_ip_i\log\frac1{p_i}\leq\sum_ip_i\log\frac1{q_i}$. -- (1)
\begin{enumerate}[(a)]
\item $D_{\min}=\min_{\q(s)}\E[d(S, \q(S))]=\min_{\q(s)}\E[\log\frac1{\q(S)}]=0$ if $\q(s)=\I\{S=s\}$.\\
$D_{\max}=\max_\q\E[d(S, \q)]=\min_\q\E[\log\frac1{\q(S)}]$.\\
$\cuz\E[\log\frac1{\q(S)}]=\suml_sP_S(s)\log\frac1{\q(S)}\overtext{(1)}\geq\suml_sP_S(s)\log\frac1{P_S(s)}=H(S)=H(\pi)$, and the equation holds when $\q(s)=P_S(s)$.\\
$\so D_{\max}=H(\pi)$.
\item $H(S|\Q)=\E_{(S, \Q)\sim P}[\log\frac1{P_{S|\Q}}]=\sum_\q P_\Q(\q)\sum_sP_{S|\Q}(s|\q)\log\frac1{P_{S|\Q}(s|\q)}\\
\overtext{(1)}\leq\sum_\q P_\Q(\q)\sum_sP_{S|\Q}(s|\q)\log\frac1{\q(s)}=\E_{(S, \Q)\sim P}\left[\log\frac1{\Q(S)}\right]$.
\item $R(D)=\inf_{(S, \Q)}\left\{I(S; \Q)\left|\E[\log\frac1{\Q(S)}]\leq D\text{ and }S\sim\pi\right.\right\}\\
=\inf_{(S, \Q)}\left\{I(S; \Q)\left|H(S|\Q)\leq\E[\log\frac1{\Q(S)}]\leq D\text{ and }S\sim\pi\right.\right\}\\
\overtext{(2)}\leq\inf_{(S, \Q)}\left\{I(S; \Q)\left|H(S|\Q)\leq D\text{ and }S\sim\pi\right.\right\}\\
\overtext{(3)}\leq\inf_{(S, \Q)}\left\{I(S; \Q)\left|H(S|\Q)\leq D\text{ and }S\sim\pi\text{ and }\Q(\hat s)=1\text{ for some }\hat s\in\mathcal{S}\right.\right\}\\
=\min_{(S, \hat S)}\left\{I(S; \hat S)\left|H(S|\hat S)\leq D\text{ and }S\sim\pi\right.\right\}$.
\item Let $\q_{\hat s}(s):=\I(s=\hat s)$.\\
Consider the distribution $\Q=\q_S$:\\
%$P_\Q(\q)=\begin{cases}P_S(s)\text{, if }\q=\q_s\text{ for some }s\\
%0\text{, otherwise}\end{cases}$.\\
The equation in (2) holds $\iff$ the equation in (1) holds $\iff\forall s, \q,\ P_{S|\Q}(s|\q)=\q(s)$, which is true because $\forall\q$ with nonzero probability, $\q=\q_{\hat s}$ for some $\hat s$, and $\q_{\hat s}(s)=\I(s=\hat s)\overset{\Q=\q_S}=P_{S|\Q}(s|\q_{\hat s})$.\\
The equation in (3) holds since $\q_{\hat s}=1$ for $\hat s\in S$.\\
$\so$ with this distribution, $R(D)=\min_{(S, \hat S)}\left\{I(S; \hat S)\left|H(S|\hat S)\leq D\text{ and }S\sim\pi\right.\right\}\\
=\min_{(S, \hat S)}\left\{H(S)-H(S|\hat S)\left|H(S|\hat S)\leq D\text{ and }S\sim\pi\right.\right\}\\
=\min_{(S, \hat S)}\left\{H(\pi)-H(S|\hat S)\left|H(S|\hat S)\leq D\text{ and }S\sim\pi\right.\right\}\\
=H(\pi)-D\overset{0\leq D\leq H(\pi)\text{ is given}}=\max(0, H(\pi)-D)$.
\end{enumerate}
\end{pr}

\begin{pr}$ $
\begin{enumerate}[(a)]
\item Let $N$ be the number of games played.\\
Given $N=n$, there is $\P[X_i=1]=\P[X_i=2]=\frac12$ for $i<n$.\\
Also, since each game loses with probability $\frac13$, there is $N\sim\mathrm{Geometric}(\frac13)$.\\
$\then Y=Z_1+Z_2+\cdots+Z_{N-1}$, where $N\sim\mathrm{Geometric}(\frac13)$ and $Z_i$ is an uniform distribution on $\{1, 2\}$.\\
Note that $Z_1, Z_2, \dots, Z_{N-1}, N$ are independent.\\
$\phi_N(s)=\frac{\frac13e^s}{1-\frac23e^s}$, and $\phi_{N-1}(s)=e^{-s}\phi_N(s)=\frac{\frac13}{1-\frac23e^s}=\frac1{3-2e^s}$.\\
By theorem 6.12 in the textbook, $\phi_Y(s)=\phi_{N-1}(\ln\phi_{Z_i}(s))=\phi_{N-1}(\ln(\frac12(e^s+e^{2s})))=\frac1{3-2\times\frac12(e^s+e^{2s})}=\frac1{3-e^s-e^{2s}}$.
%\frac{\frac13\times\frac12(1+e^s)}{1-\frac23\times\frac12(e^s+e^{2s})}=\frac{1+e^s}{6-2e^s-2e^{2s}}$.
\item $\E[Y]=\phi_Y'(0)\\
=\left.\left(-\frac{-e^s-2e^{2s}}{(3-e^s-e^{2s})^2}\right)\right|_{s=0}\\
=\left.\left(\frac{e^s+2e^{2s}}{(3-e^s-e^{2s})^2}\right)\right|_{s=0}\\
=\frac3{1^2}=3$.\\
$\E[Y^2]=\phi_Y''(0)\\
=\left.\left(\frac{e^s+4e^{2s}}{(3-e^s-e^{2s})^2}-2\cdot\frac{(e^s+2e^{2s})(-e^s-2e^{2s})}{(3-e^s-e^{2s})^3}\right)\right|_{s=0}\\
=\frac5{1^2}-2\cdot\frac{3\times(-3)}{1^3}=5+18=23$.\\
$\Var(Y)=\E[Y^2]-(\E[Y])^2=23-3^2=14$.
\end{enumerate}
\end{pr}

\begin{pr}
Suppose that the $a_1, a_2, \dots, a_{2n}$ are $2n$ points chosen uniform randomly in $[0, 1]$.\\
WLOG suppose that the intervals of the first slot are $[a_1, a_1+x], [a_2, a_2+y]\pmod1$, and WLOG suppose that $a_2-a_1\pmod1\leq\frac12$.\\
$\P[x+y<\frac1{4n\sqrt n}]<\P[x<\frac1{4n\sqrt n}\land y<\frac1{4n\sqrt n}]$.\\
Let $A_1:=[a_1, a_1+\frac1{4n\sqrt n}), A_2:=[a_2, a_2+\frac1{4n\sqrt n})$.\\
There are two cases that $x<\frac1{4n\sqrt n}\land y<\frac1{4n\sqrt n}$.\\
Case 1: $a_2\in A_1$.\\
$x$ is guaranteed to $<\frac1{4n\sqrt n}$ in this case.\\
The probability that case 1 happens $=\frac1{4n\sqrt n}$.\\
$\P[y<\frac1{4n\sqrt n}]=\P[a_3\in A_2\lor\cdots\lor a_{2n}\in A_2]\leq\suml_{i=3}^{2n}\P[a_i\in A_2]=\suml_{i=3}^{2n}\frac1{4n\sqrt n}=\frac{2n-2}{4n\sqrt n}\leq\frac{2n}{4n\sqrt n}=\frac1{2\sqrt n}$.\\
Case 2: $a_2\notin A_1$.\\
The probability that case 2 happens $=1-\frac1{4n\sqrt n}$.\\
By the assumption that $a_2-a_1\pmod1\leq\frac12$, there is $A_1\cap A_2=\emptyset$.\\
$\P[x<\frac1{4n\sqrt n}\land y<\frac1{4n\sqrt n}]=\P[\bigcup_{i=3}^{2n}\bigcup_{j=i+1}^{2n}(a_i\in A_1\land a_j\in A_2)\lor(a_i\in A_2\land a_j\in A_1)]\leq\suml_{i=3}^{2n}\suml_{j=i+1}^{2n}\P[(a_i\in A_1\land a_j\in A_2)\lor(a_i\in A_2\land a_j\in A_1)]=\suml_{i=3}^{2n}\suml_{j=i+1}^{2n}2\times\frac1{4n\sqrt n}\times\frac1{4n\sqrt n}=\binom{2n-2}2\times2\times\frac1{16n^3}\leq\frac{4n^2}{16n^3}=\frac1{4n}$.\\
$\so\P[x+y<\frac1{4n\sqrt n}]\leq\frac1{4n\sqrt n}\times\frac1{2\sqrt n}+(1-\frac1{4n\sqrt n})\times\frac1{4n}\leq\frac1{8n^2}+\frac1{4n}\leq\frac1{2n}$.\\
$\P[$ the size of the smallest interval $<\frac1{4n\sqrt n}]\\
=\P[\bigcup_{i=1}^n($ the size of the interval of the $i$-th slot $<\frac1{4n\sqrt n})]\\
\leq\suml_{i=1}^n\P[$ the size of the interval of the $i$-th slot $<\frac1{4n\sqrt n}]\\
\leq\suml_{i=1}^n\frac1{2n}=\frac12$.\\
$\so$ the size of the smallest interval is at least $\frac1{4n\sqrt n}$ with probability $\geq1-\frac12=\frac12=\Omega(1)$.
\end{pr}

\begin{pr}
The probability that Dr. Jones purchased a new umbrella on a day is the chance of raining and Dr. Jones visiting the library and the umbrella being stolen, which is $0.5\times0.8\times0.25=0.1$.\\
The probability that the $10$-th time getting stolen happens right on the $20$-th day is the probability that Dr. Jones got stolen for exactly $9$ times in the $1$-st to $19$-th day, and got stolen in the $20$-th day, which is $\binom{19}90.9^{10}0.1^9\times0.1=\binom{19}90.09^{10}$.
\end{pr}

\newpage


\end{document}
