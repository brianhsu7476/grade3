%\documentclass[12pt, a4paper,oneside]{book}
\documentclass[12pt, a4paper, twocolumn]{article}
\setlength{\columnseprule}{0.4pt}
\usepackage[left=1cm, right=1cm, top=2cm, bottom=2cm]{geometry}
\usepackage{amsmath,amsthm,amssymb,mathrsfs}
%\usepackage{yhmath}
\usepackage{graphicx}
\usepackage{fontspec}
\usepackage{type1cm}
\usepackage{titlesec}
\usepackage{titling}
\usepackage{fancyhdr}
\usepackage{tabularx}
\usepackage[square, comma, numbers, super, sort&compress]{natbib}
\usepackage[unicode=true, pdfborder={0 0 0}, bookmarksdepth=-1]{hyperref}
\usepackage[noabbrev,capitalize]{cleveref}
\usepackage[usenames, dvipsnames]{color}
\usepackage{enumerate}
\usepackage{tikz-cd}
\usepackage{centernot}
\usepackage{cases}
\usepackage{indentfirst}
\usepackage{tkz-berge}
\usepackage{tkz-graph}
\usetikzlibrary{arrows, petri, topaths}

%\usepackage{minted, listings}
\usepackage{listings}
\lstset{
    basicstyle=\ttfamily,
    breaklines,
    backgroundcolor = \color{lightgray},
    numbers=left
}
%\newminted[cpp]{c++}{
    %frame=single,
    %frame=lines,
    %framesep=0mm,
    %breaklines,
    %linenos,
    %mathescape,
    %xleftmargin=1cm,
    %baselinestretch=1.6,
    %style=manni,
    %fontsize=,
%}

\usepackage{xeCJK}
    \setCJKmainfont[AutoFakeBold=6,AutoFakeSlant=.2]{AR PL KaitiM Big5}
    \XeTeXlinebreaklocale "zh"
    \XeTeXlinebreakskip = 0pt plus 1pt

\usepackage{tocloft}
    \renewcommand{\cftsecleader}{\cftdotfill{\cftdotsep}}
    \setcounter{tocdepth}{1}

\renewcommand{\baselinestretch}{1.5}

\setlength{\headheight}{15pt}
\setlength{\droptitle}{-1.5cm}

\renewcommand{\contentsname}{目錄}
\renewcommand{\bibsection}{\bf\Large 參考資料\normalsize\rm}
\renewcommand{\bibname}{\bf 參考資料}

\hyphenation{word list}

\newcommand{\A}{\mathcal{A}}
\newcommand{\B}{\mathcal{B}}
\renewcommand{\b}{\overline}
\newcommand{\C}{\mathbb{C}}
\newcommand{\D}{\mathrm{D}}
\newcommand{\E}{\mathrm{E}}
\renewcommand{\H}{\mathrm{H}}
\newcommand{\I}{\mathbb{I}}
\newcommand{\N}{\mathbb{N}}
\renewcommand{\P}{\mathrm{Pr}}
\newcommand{\Q}{\mathbb{Q}}
\newcommand{\R}{\mathbb{R}}
\newcommand{\V}{\Vertex}
\newcommand{\Z}{\mathbb{Z}}
\newcommand{\simple}{\SetVertexSimple[MinSize=8pt]}
\newcommand{\pictures}{\includegraphics[height=6cm]}
\newcommand{\directed}{\tikzstyle{EdgeStyle}=[post]}
\newcommand{\bidirected}{\tikzstyle{EdgeStyle}=[pre and post]}
\newcommand{\degree}{^\circ}
\newcommand{\Arc}[1]{\wideparen{{#1}}}
\newcommand{\lin}{\overleftrightarrow}
\newcommand{\ray}{\overrightarrow}
\newcommand{\seg}{\overline}
\newcommand{\ang}{\angle}
\newcommand{\inv}{^{-1}}
\newcommand{\dang}{\measuredangle}
\newcommand{\then}{\Rightarrow}
\newcommand{\cuz}{\because}
\newcommand{\so}{\therefore}
%\newcommand{\D}[3][2]{\left|\begin{array}{cc}#2&#3\\\b{#2}&\b{#3}\end{array}\right|}
\newcommand{\spc}{\;\;\;\;\;\;}
\newcommand{\st}{\text{ s.t. }}
\newcommand{\claim}{\textit{Claim}}
\renewcommand{\ll}{\longleftarrow}
\newcommand{\lr}{\longrightarrow}
\renewcommand{\mod}{\mathrm{\ mod\ }}
\newcommand{\dep}{\mathrm{depth}}
\newcommand{\contradict}{\rightarrow\leftarrow}
\newcommand{\deter}[9]{\left|\begin{array}{ccc}#1&#2&#3\\#4&#5&#6\\#7&#8&#9\end{array}\right|}
\newcommand{\psim}{\stackrel{+}{\sim}}
\newcommand{\msim}{\stackrel{-}{\sim}}
\newcommand{\ma}{\measuredangle}
\newcommand{\parato}{\,\|\,}
\DeclareMathOperator{\lcm}{lcm}
\DeclareMathOperator{\Spec}{Spec}
\newcommand{\erf}{\mathrm{erf}}
\newcommand{\Ker}{\mathrm{Ker}}
\renewcommand{\Im}{\mathrm{Im}}
\newcommand{\Mn}{\mathrm{Mn}}
\renewcommand{\dim}{\mathrm{dim}}
\newcommand{\rank}{\mathrm{rank}}
\renewcommand{\matrix}[1]{\begin{pmatrix}#1\end{pmatrix}}
\newcommand{\nmatrix}[1]{\begin{array}{ccccccccccccccccccccccccccc}#1\end{array}}
\newcommand{\nullity}{\mathrm{nullity}}
\newcommand{\adj}{\mathrm{adj}}
\newcommand{\Id}{\mathrm{Id}}
\newcommand{\Tr}{\mathrm{Tr}}
\newcommand{\ch}{\mathrm{ch}}
\newcommand{\Span}{\mathrm{span}}
\newcommand{\inn}[2]{\left\langle#1,#2\right\rangle}
\newcommand{\abs}[1]{\left\|#1\right\|}
\newcommand{\rfeq}[1]{\textbf{Equation (\ref{#1})}}
\newcommand{\rflm}[1]{\textbf{Lemma (\ref{#1})}}
\newcommand{\overtext}[1]{\overset{\text{#1}}}
\newcommand{\Proj}{\mathrm{Proj}}
\newcommand{\suml}{\sum\limits}
\newcommand{\prodl}{\prod\limits}
\newcommand{\capl}{\bigcap\limits}
\newcommand{\cupl}{\bigcup\limits}
\newcommand{\Null}{\mathrm{Null}}
\renewcommand{\sp}[1]{\abs{#1}_{\mathrm{sp}}}
\newcommand{\diag}{\mathrm{diag}}
\newcommand{\Ber}{\mathrm{Ber}}
\def\bbinom#1#2{\ensuremath{\left(\kern-.3em\left(\genfrac{}{}{0pt}{}{#1}{#2}\right)\kern-.3em\right)}}
\newcommand{\subj}{\mathrm{subject to: }}


\newtheoremstyle{mystyle}
  {15pt}{15pt}
  {}
  {}
  {\bf}
  {.}
  {1em}
  {}

\theoremstyle{mystyle}
\newtheorem{thm}{Theorem}[]
\newtheorem{thr}[thm]{引理}%alias
\newtheorem{ex}[thm]{Example}
\newtheorem{pr}{Problem}[]
\newtheorem{sbpr}{Subproblem}[pr]
\newtheorem{df}[thm]{Definition}
\newtheorem{pp}[thm]{Proposition}
\newtheorem{co}[thm]{Corollary}
\newtheorem{for}[thm]{Formula}
\newtheorem{lm}{Lemma}[pr]
\newtheorem{nota}[thm]{Notation}
\newtheorem{cl}{Claim}[pr]
\newtheorem{rmk}{Remark}[]
\numberwithin{equation}{pr}

\newtheoremstyle{pfstyle}
  {0pt}{3pt}
  {}
  {}
  {\it}
  {.}
  {1em}
  {}

\theoremstyle{pfstyle}
%\newtheorem{cl}{Claim}[pr]
\newtheorem*{pf}{pf}
\newtheorem*{tht}{想法}
%\newtheorem*{rmk}{\bf 註}

\newcommand{\real}{\mathbb{R}}
\newcommand{\rational}{\mathbb{Q}}
\newcommand{\complex}{\mathbb{C}}
\newcommand{\integer}{\mathbb{Z}}
\newcommand{\nat}{\mathbb{N}}

%\newenvironment{solution}
%    {\renewcommand\qedsymbol{$\blacksquare$}\begin{proof}[Solution]}
%    {\end{proof}}
%\newenvironment{sketch}
%    {\begin{proof}[Sketch of Proof]}
%    {\end{proof}}
%\ExplSyntaxOn
%\NewDocumentCommand{\cycle}{ O{\;} m }
% {
%  (
%  \alec_cycle:nn { #1 } { #2 }
%  )
% }

%\seq_new:N \l_alec_cycle_seq
%\cs_new_protected:Npn \alec_cycle:nn #1 #2
% {
%  \seq_set_split:Nnn \l_alec_cycle_seq { , } { #2 }
%  \seq_use:Nn \l_alec_cycle_seq { #1 }
% }
%\ExplSyntaxOff

\renewcommand\qedsymbol{$\blacksquare$}

\everymath{\displaystyle}

\begin{document}

    %\settowidth{\parindent}{一二}
%\setlength{\parindent}{24pt}
\setlength{\parindent}{0pt}

\pagenumbering{gobble}
\title{Graph Theory HW4}
\author{許博翔 B10902085\\
Teammate: 黃芊禕 B10902029}
\date{\today}
\maketitle
%\tableofcontents
\thispagestyle{empty}
\pagenumbering{arabic}
\setcounter{page}{1}
\pagestyle{fancy}

\renewcommand{\sectionmark}[1]{\markright{#1}}
\renewcommand{\subsectionmark}[1]{}

\lhead{\thetitle}
\chead{}
\rhead{}
\lfoot{Author:~\theauthor}
\cfoot{}
\rfoot{\thepage}
\renewcommand{\headrulewidth}{0.4pt}
\renewcommand{\footrulewidth}{0.4pt}

%\setcounter{chapter}{1}
%\setcounter{section}{-1}

%\mainmatter
\begin{pr}$ $
\begin{enumerate}[(a)]
\item $\overline{m_1}=\min_{t\in T}\max_{s\in S}\pi_1(s, t)=\min(1, 2)=1$.\\
$\overline{m_2}=\min_{s\in S}\max_{t\in T}\pi_2(s, t)=\min(2, 1)=1$.\\
$\so$ by Folk theorem, the payoffs they could achieve in a Nash equilibrium is larger than or equal to $(\overline{m_1}, \overline{m_2})=(1, 1)$.\\
Therefore, the payoffs they could achieve in a Nash equilibrium is the colored region in the following graph:\\
\includegraphics[width=15cm]{1a.JPG}
% picture
\item The upper one is the row player's strategy, while the bottom one is the columns player's strategy. The strategy is to play $(s_1, s_1), (s_1, s_1), (s_2, s_2)$ cyclically. Since if one of them change the strategy, their long-run average payoff will decrease to $1$, no one can change the strategy to achieve a better payoff. Therefore, the following strategy is a Nash equilibrium.\\
\includegraphics[width=15cm]{1b.JPG}
% picture
\item Suppose that $(p, 1-p)$ is the row player's mixed strategy.\\
$\underline{v_1}=\max_p\min_{t\in T}\E(\pi_1(s, t))=\max_p\min(2p, 1-p)=\frac23$, where $p=\frac13$.\\
Similarly, $\underline{v_2}=\frac23$.\\
$\so$ by Folk theorem, the payoffs they could achieve in a Nash equilibrium is larger than or equal to $(\frac23, \frac23)$.\\
Therefore, the payoffs they could achieve in a Nash equilibrium is the colored region in the following graph:\\
\includegraphics[width=15cm]{1c.JPG}
% picture
\end{enumerate}
\end{pr}

\begin{pr}
\texttt{ADD successor to closed} should not be at the first time \texttt{successor} is inserted to fringe. Instead, it should be at the first time \texttt{successor} is removed from fringe. Reason: The first time that \texttt{successor} is inserted may not have the highest priority. If \texttt{ADD successor to closed} is added at the first time \texttt{successor} is inserted to fringe, then the solution it finds may not be the optimal one.
\end{pr}

\setcounter{pr}{5}
\begin{pr}
Consider a spanning tree $T$ of $G$.\\
If $d_T(u, v)\leq3$, then $d_G(u, v)\leq d_T(u, v)\leq3$, and therefore $uv\in E(G^3)$.\\
Suppose that $|V(G)|=n$.\\
Let's prove that there exists a sequence of vertices $v_0(=v_n), v_1, v_2, \dots, v_n$ where $V(G)=\{v_1, \dots, v_n\}$ and $d_T(v_i, v_{i+1})\leq3,\ \forall0\leq i\leq n-1$.\\
If such sequence exists, since "$d_T(u, v)\leq3\then d_G(u, v)\leq d_T(u, v)\leq3\then uv\in E(G^3)$", $G^3$ is Hamiltonian.\\
The exists of such sequence:\\
Choose an arbitrary vertex in $T$ to be a root, and colors all vertices with even depth in black, all vertices with odd depth in white.\\
Since the depth of a neighbor $u$ of a vertex $v$ is exactly greater or less than that of $v$ by $1$, all adjacent neighbors have different colors.\\
Let's use the following algorithm to find the sequence.\\
Use DFS (depth-first search) to visit all the vertex.\\
We can treat DFS as a walk on the tree, and let $p$ be the walk.\\
Claim: mark the first appearance of each black vertex in $p$, and the last appearance of each white vertex in $p$, and $v_i:=$ the $(i+1)$-th ($1$-base) marked vertex in $p$ for $0\leq i\leq n-1$, $v_n:=v_0$. This sequence satisfies $d_T(v_i, v_{i+1})\leq3,\ \forall0\leq i\leq n-1$.\\ Furthermore, the depth of $v_{n-1}$ is $1$.\\
Proof: we can do induction on the number of vertices of the tree (I'll prove this claim holds for arbitrary tree).\\
For the tree with one vertex, it's trivial.\\
For the trees with more than one vertex, let's $r$ be the root of the tree, and $s_1, s_2, \dots, s_m$ be the children of $r$.\\
By the induction hypothesis, each of the $m$ subtree of $s_1, s_2, \dots, s_m$ has such sequence, and suppose that the sequence of the subtree of $s_i$ is $v_{i0}, v_{i1}, \dots, v_{in_i}$.\\
Since reversing the colors of all vertices (can) reverse the order of the sequence (by reversing the order of the children of all vertices being visited in DFS), the sequence is $r, v_{1(n_1-1)}, \dots, v_{11}, v_{10}, v_{2(n_2-1)}, \dots, \dots, v_{m(n_m-1)}, \dots, v_{m1}, v_{m0}, r$.\\
By the induction hypothesis, the depth of $v_{i(n_i-1)}$ is $2$ and the depth of $v_{i0}$ is $1$, $d_T(r, v_{1(n-1})=2, d_T(v_{m0}, r)=1, d_T(v_{i0}, v_{{i+1}(n_{i+1}-1)})=3$ for all $1\leq i\leq m-1$.\\
$\so$ the sequence satisfies condition in the claim.\\
$\so$ by induction, such sequence exists for all trees $T$.\\
$\so G^3$ is Hamiltonian.
\end{pr}

\newpage


\end{document}
