%\begin{df}
%Define $\binom nk:=0$ when $k<0$ or $k>n$.
%\end{df}

\begin{lm}
\label{lm0}
Let $f(p, k):=\suml_{n=k}^\infty\binom nkp^n$.\\
Then $f(p, k)=\frac1{1-p}(\frac p{1-p})^k$.
\begin{proof}
Let's prove by induction on $k$.\\
For $k=0, f(p, 0)=\suml_{n=0}^\infty\binom n0p^n=\frac1{1-p}$.\\
Suppose for $k=k'$, $f(p, k')=\frac1{1-p}(\frac p{1-p})^{k'}$.\\
For $k=k'+1$, $f(p, k'+1)=\suml_{n=k'+1}^\infty\binom n{k'+1}p^n=\suml_{n=k'+1}^\infty(\binom{n-1}{k'+1}+\binom{n-1}{k'})p^n=\suml_{n=k'+1}^\infty(\binom n{k'+1}p^n+\binom{n-1}{k'}p^{n-1})p=pf(p, k'+1)+\suml_{n=k'}^\infty\binom n{k'}p^{n-1}p=p(f(p, k'+1)+f(p, k'))$.\\
$\then(1-p)f(p, k'+1)=pf(p, k')$.\\
$\then f(p, k'+1)=\frac p{1-p}f(p, k')=\frac1{1-p}(\frac p{1-p})^{k'+1}$.\\
$\so$ by induction, \rflm{lm0} holds.
\end{proof}
\end{lm}
