\begin{pr}
First, $X\sim\mathrm{Unif}[0, \frac d2], \Theta\sim\mathrm{Unif}[0, \frac\pi2]$.\\
$\then f_X(x)=\begin{cases}
\frac2d\text{, if }0\leq x\leq\frac d2\\
0\text{, otherwise}
\end{cases}, f_\Theta(\theta)=\begin{cases}
\frac2\pi\text{, if }0\leq\theta\leq\frac\pi2\\
0\text{, otherwise}
\end{cases}$.\\
The needle intersects one of the lines $\iff X\leq\frac l2\sin\theta$.\\
$\so$ the probability that the needle will intersect one of the lines $=\P[X\leq\frac l2\sin\theta]$.\\
Note that $l<d$, so the upperbound of $X$ in the following integral is $\min(\frac l2\sin\theta, \frac d2)=\frac l2\sin\theta$.\\
And $X, \Theta$ are independent, so their joint pdf $f_{X, \Theta}(x, \theta)=f_X(x)f_\Theta(\theta)$.\\
$\then\P[X\leq\frac l2\sin\theta]=\int_0^{\frac\pi2}\int_0^{\frac l2\sin\theta}f_X(x)f_\Theta(\theta)dxd\theta=\int_0^{\frac\pi2}\frac l2\sin\theta\frac2d\frac2\pi d\theta=\frac{2l}{d\pi}$.\\
This experiment can get $p$: the approximated value of the probability that the needle will intersect one of the lines when the needle is dropped for sufficient large number of times.\\
And one can approximate $\pi\approx\frac{2l}{dp}$.
\end{pr}
