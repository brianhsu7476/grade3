\begin{multicols}{3}
金融科技使得金融服務具有\\
  效率性 \\
  普及性 \\
正確!\\
  皆是 \\
 \\
問題 2\\
5 / 5 分\\
目前CFA 考慮的 Fintech 範圍不包含哪項?\\
  data analytics \\
  robo advisor \\
正確!\\
  crowd funding \\
 \\
問題 3\\
5 / 5 分\\
以下哪些屬於另類資料 (alternative data),與傳統金融的結構化資料不同?\\
  社群留言 \\
  衛星影像 \\
  大氣微粒子 \\
正確!\\
  以上皆是 \\
 \\
問題 4\\
5 / 5 分\\
另類資料的哪些特色使得數據分析較為困難?\\
  數據量大 \\
  複雜度高 \\
  資料清洗不易 \\
正確!\\
  以上皆是 \\
 \\
問題 5\\
5 / 5 分\\
「以人為本的可信任AI」是為了回應AI預測可能造成某些特定的風險的誤判風險,具體作法包括:\\
  謹慎開發或使用AI這類強大的工具以避免偏誤 \\
  將風險就使用場合、可能造成的損害,以及損害發生的機率等做分級 \\
  聚焦高度風險的監理,例如信用風險等,以避免過度監管 \\
正確!\\
  以上皆是 \\
 \\
問題 6\\
0 / 5 分\\
平台經濟模式的目標是?\\
正確答案\\
  金融生態圈 \\
您已回答\\
  數位轉型 \\
  跨域人才培養 \\
  AI 技術進步 \\
 \\
問題 7\\
5 / 5 分\\
機器人理專屬於金融六大功能的哪一項?\\
正確!\\
  投資管理 \\
  籌資 \\
  支付 \\
 \\
問題 8\\
5 / 5 分\\
機器人理專主要交易何種標的資產?\\
  股票 \\
  基金 \\
正確!\\
  ETF \\
 \\
問題 9\\
5 / 5 分\\
以下那一項不是現行機器人理專缺點\\
  無面對面服務 \\
正確!\\
  低波動 \\
  缺乏客製化 \\
 \\
問題 10\\
0 / 5 分\\
機器人理專提供的資產配置是\\
您已回答\\
  靜態的 \\
正確答案\\
  動態的 \\
  高波動 \\
  費用率高 \\
 \\
問題 11\\
5 / 5 分\\
在每個風險下找出最佳的投資組合,這些投資組合預期風險與預期報酬的點集合可畫出的曲線是\\
  無異曲線 \\
  資本市場線 \\
正確!\\
  效率前緣 \\
  證券市場線 \\
 \\
問題 12\\
5 / 5 分\\
假設投資組合A和B有相同的平均報酬,相同的報酬標準差,但投資組合A相較於投資組合B有較低的貝塔值(beta)。以夏普比例(Sharpe ratio)而言,投資組合A的表現:\\
  優於投資組合B的表現 \\
正確!\\
  相同於投資組合B的表現 \\
  劣於投資組合B的表現 \\
  資訊不足以判斷此問題 \\
 \\
問題 13\\
5 / 5 分\\
ETF 具有以下哪些特質?\\
  股票 \\
  基金 \\
正確!\\
  皆有 \\
 \\
問題 14\\
5 / 5 分\\
目前世界最大 AUM 的 ETF 是?\\
正確!\\
  SPY \\
  VOO \\
  SSO \\
 \\
問題 15\\
0 / 5 分\\
槓反型ETF以操作哪種資產為主?\\
正確答案\\
  期貨 \\
您已回答\\
  股票 \\
  債券 \\
 \\
問題 16\\
5 / 5 分\\
ETF 的風險包括\\
  交易對手 \\
  稅務 \\
正確!\\
  皆是 \\
 \\
問題 17\\
5 / 5 分\\
衡量期貨ETF 績效的方式有\\
  效率性 (efficiency) \\
  交易性 (tradability) \\
  適配性 (fit) \\
正確!\\
  皆可 \\
 \\
問題 18\\
0 / 5 分\\
若是反向兩倍ETF所連結的指數第一天上漲10%,第二天下跌10%,此ETF累積之報酬率為?\\
正確答案\\
  -4% \\
  -2% \\
  0% \\
您已回答\\
  2% \\
 \\
問題 19\\
5 / 5 分\\
微保險的特色包括\\
  低保險給付 \\
  快速理賠 \\
正確!\\
  皆是 \\
 \\
問題 20\\
5 / 5 分\\
物聯網在保險上的應用流程不包括\\
  使用感測器 \\
  大數據分析 \\
  風險評估 \\
正確!\\
  理賠 \\
  \\
外匯遠期契約可以視為一種線性的商品\\
正確!\\
  正確 \\
  錯誤 \\
 \\
問題 2\\
0 / 5 分\\
理論上股指期貨價格應比現貨價格高\\
正確答案\\
  正確 \\
您已回答\\
  錯誤 \\
 \\
問題 3\\
5 / 5 分\\
金融衍生品只有歐式買權與賣權兩種形式\\
  正確 \\
正確!\\
  錯誤 \\
 \\
問題 4\\
5 / 5 分\\
選擇權市場與債券市場可以互相套利\\
正確!\\
  正確 \\
  錯誤 \\
 \\
問題 5\\
5 / 5 分\\
二元樹模型是離散時間下的選擇權定價模型\\
正確!\\
  正確 \\
  錯誤 \\
 \\
問題 6\\
5 / 5 分\\
何者是金融工程領域中所榮獲的諾貝爾獎\\
  投資組合理論 \\
  選權權定價 \\
正確!\\
  以上皆是 \\
 \\
問題 7\\
0 / 5 分\\
譽為數理金融之父的是\\
您已回答\\
  Black-Scholes \\
  Richard Feynman \\
正確答案\\
  Louis Bachelier \\
 \\
問題 8\\
0 / 5 分\\
金融工程的典範機構為\\
正確答案\\
  CBOE \\
  IMF \\
您已回答\\
  WEF \\
 \\
問題 9\\
5 / 5 分\\
期貨採取哪種交易方式以規避違約風險\\
  保證金制度 \\
  每日結算 \\
正確!\\
  以上皆是 \\
 \\
問題 10\\
5 / 5 分\\
期貨市場不具備哪一項功能\\
  投機 \\
正確!\\
  穩定金融市場 \\
  避險 \\
 \\
問題 11\\
5 / 5 分\\
利用哪種合約可以轉換公司間的資產或負債\\
正確!\\
  交換 (swap) \\
  遠期 (forward) \\
  期貨 (futures) \\
 \\
問題 12\\
5 / 5 分\\
選擇權的報酬函數主要呈現\\
  線性 \\
正確!\\
  非線性 \\
  皆可 \\
 \\
問題 13\\
5 / 5 分\\
哪種契約規範了買方有權利,但非義務,可以在某固定的到期日 (maturity, expiration date) ,以一個預先決定好了的履約價格 ,來買入標的資產\\
正確!\\
  買權 \\
  賣權 \\
  期貨 \\
 \\
問題 14\\
0 / 5 分\\
選擇權契約的賣方在取得了權利金之後,如何支付未來可能的報酬問題,屬於哪類\\
您已回答\\
  定價 \\
正確答案\\
  避險 \\
  套利 \\
 \\
問題 15\\
5 / 5 分\\
何者交易行為造成閃崩\\
  套利 \\
  配對 \\
正確!\\
  高頻 \\
 \\
問題 16\\
5 / 5 分\\
以下何種數學模型適合股價\\
  布朗運動(Brownian motion) \\
正確!\\
  幾何布朗運動 (Geometric Brownian motion) \\
  Poisson processes \\
 \\
問題 17\\
0 / 5 分\\
風險中立評價法主要是基於哪個假設\\
您已回答\\
  最佳複製 \\
正確答案\\
  完美複製 \\
  最小變異 \\
 \\
問題 18\\
5 / 5 分\\
選擇權定價必須在哪種機率測度(probability measure)之下\\
正確!\\
  風險中立機率測度 (Risk-neutral probability measure) \\
  歷史機率測度 (Historical probability measure) \\
  皆可 \\
 \\
問題 19\\
5 / 5 分\\
最類似於Black-Scholes pricing PDE是哪一種\\
正確!\\
  熱傳導方程式 (heat equation) \\
  歷史機率測度 (Historical probability measure) \\
  皆可 \\
 \\
問題 20\\
5 / 5 分\\
Black-Scholes option pricing theory 中哪個量最常用來衡量選擇權的風險\\
  履約價 (Strike price) \\
  到期日 (Expiration date) \\
正確!\\
  波動率 (Volatility) \\
 \\
問題 21\\
0 / 10 分\\
按單期二元定價模型,若期初股價為1元,一季內上漲與下跌幅度皆為10%,履約價亦為1元,無風險利率為0,一季後到期的歐式賣權的理論價格應為? (四捨五入至小數點後第二位 Ex. 答案若為 0.123,則填 0.12;若為 0.1,就填 0.10 即可)\\
 \\
您已回答\\
1.11\\
正確答案\\
0.05 \\
\end{multicols}
