\setcounter{pr}{0}
\begin{pr}$ $
\begin{enumerate}[(a)]
\item If two distinct edges have a common vertex and are colored in different colors, we call them a \textbf{coloring angle}.\\
In a non-monochromatic triangle, the multiset of the color of the edges is $\{R, R, B\}$ or $\{B, B, R\}$.\\
$\then$ there is exactly one vertex that is incident to the edges of the same color in a non-monochromatic triangle.\\
$\then$ there are exactly two vertices that are incident to edges of different colors in a non-monochromatic triangle.\\
$\then$ there are exactly two coloring angles of a non-monochromatic triangle.\\
Also, for any two edges $uv, uw$ that have a common vertex, there is a unique triangle $uvw$ that uses both of the edges.\\
$\so$ every coloring angle is contained in exactly one non-monochromatic triangle.\\
$\so$ the number of non-monochromatic triangle $=\frac12\times($ the number of coloring angle $)$.\\
Every coloring angle has a unique common vertex, so we can count the number of coloring angles by vertices.\\
The number of coloring angles having the common vertex $v$ is the number of ways to select two edges incident to $v$ with different colors, which is $r_v(n-1-r_v)$.\\
$\so$ the number of non-monochromatic triangles $=\frac12\times($ the number of coloring angle $)=\frac12\sum_{i=1}^nr_i(n-1-r_i)$.\\
$\then$ the number of monochromatic triangles $=\binom n3-\frac12\sum_{i=1}^nr_i(n-1-r_i)$.
\item By AM-GM inequality, $\sqrt{r_i(n-1-r_i)}\leq\frac{n-1}2$.\\
$\then\binom n3-\frac12\suml_{i=1}^nr_i(n-1-r_i)\geq\binom n3-\frac12\suml_{i=1}^n(\frac{n-1}2)^2=\binom n3-\frac18n(n-1)^2=\frac16(n^3-3n^2+2n)-\frac18(n^3-2n^2+n)=\frac1{24}n^3-\frac14n^2+\frac5{24}n=(\frac1{24}-\frac1{4n}+\frac5{24n^2})n^3=(\frac1{24}-o(1))n^3$.
\end{enumerate}
\end{pr}
