\begin{pr}$ $
\begin{enumerate}[(a)]
\item For every coloring $c:2^{[n]}\to[r]$, consider the following graph $G$ and the corresponding edge-coloring $d$:\\
$G=\left([n], \binom{[n]}2\right)$, for all $uv\in E$, WLOG suppose that $u<v$, $d(uv):=c([v-1]\setminus[u-1])$.\\
By what we learned in class, $R(\underbrace{3, 3, \dots, 3}_{r})$ is finite.\\
$\so$ for $n$ large enough, for every edge-coloring of $G$, there exists a monochromatic triangle $uvw$.\\
Let $n$ be large enough, and $uvw$ be a monochromatic triangle, WLOG suppose that $u<v<w$.\\
$\then d(uv)=d(vw)=d(wu)$.\\
$\then c(\{u, u+1, \dots, v-1\})=c(\{v, v+1, \dots, w-1\})=c(\{u, u+1, \dots, w-1\})$.\\
$\so X=\{u, u+1, \dots, v-1\}, Y=\{v, v+1, \dots, w-1\}$ satisfy that $X, Y, X\cup Y$ receive the same color.

\item Suppose that the numbers in $\N$ are colored with $r$ colors, and let $c:\N\to[r]$ be a coloring.\\
Consider the following graph $G_n$ and the corresponding edge-coloring $d$:\\
$G_n:=\left([n], \binom{[n]}2\right)$, for all $uv\in E$, $d(uv):=c(|u-v|)$.\\
By what we learned in class, $R(\underbrace{3, 3, \dots, 3}_{r})$ is finite.\\
$\so\exists n$ s.t. for every edge-coloring of $G_n$, there exists a monochromatic triangle $uvw$.\\
WLOG suppose that $u<v<w$.\\
$\then c(v-u)=c(w-v)=c(w-u)$.\\
Let $x=v-u, y=w-v, z=w-u$, and they are monochromatic satisfying $x+y=z$.
\end{enumerate}
\end{pr}
