\section{Shell Scripting}

這題的目標是要寫出一個程式\texttt{compare.sh},比較兩個檔案或兩個目錄的差別。

\subsection{參數檢查}

\begin{lstlisting}
usage: ./compare.sh [OPTION] <PATH A> <PATH B>
options:
-a: compare hidden files instead of ignoring them
-h: output information about compare.sh
-l: treat symlinks as files instead of ignoring them
-n <NAME>: compare only files whose paths contain <NAME> as a substring
-r: compare directories recursively
\end{lstlisting}

指令:
\begin{enumerate}
\item 所有symlink皆視為非目錄。
\item 如果\texttt{[OPTION]}中有\texttt{-r},則\texttt{<PATH A>}、\texttt{<PATH B>}都要是目錄,否則都不能是目錄。
\item 如果\texttt{[OPTION]}中有\texttt{-a}或\texttt{-n},則必需有\texttt{-r}。
\item \texttt{<PATH A>}、\texttt{<PATH B>}需出現在所有\texttt{[OPTION]}的後面。
\item 不能有方框中沒有定義的參數。
\item 所有方框中的參數皆需照其定義的方式出現,例如\texttt{-n}後面要接額外的字串\texttt{<NAME>},而\texttt{-l}後面不能有額外的字串。
\end{enumerate}

如違反上述規定,或是\texttt{[OPTION]}中有\texttt{-h},則輸出方框中的內容並結束執行,否則不能輸出方框中的內容並且繼續執行程式。這個小題只要能在對的時機輸出方框中的內容即可得分。

\subsection{比較檔案}

這個小題的指令皆為
\begin{lstlisting}
./compare.sh <PATH A> <PATH B>
\end{lstlisting}
這個型式,在判斷參數合法之後,如兩檔案內容一樣,則什麼都不需要輸出,否則需輸出
\begin{lstlisting}
changed ${x}%
\end{lstlisting}
其中\texttt{\$\{x\}}計算方法如下:
\begin{enumerate}
\item 如果\texttt{<PATH A>}或\texttt{<PATH B>}並非文字檔(即兩者使用\texttt{diff}指令比較之後會輸出\texttt{binary files <PATH A> <PATH B> differ}),則\texttt{\$\{x\}}$:=100$。
\item 否則設要在\texttt{<PATH A>}中刪除$a$行保留$c$行插入$b$行會變成\texttt{<PATH B>},\texttt{\$\{x\}}$:=\lfloor\frac{100\max(a, b)}{\max(a, b)+c}\rfloor$的最小可能值。
\end{enumerate}
舉例來說設兩個檔案的內容分別為
\begin{lstlisting}
hello
world
and
hello
kitty
\end{lstlisting}
以及
\begin{lstlisting}
hi
world
and
kitty
\end{lstlisting}
可以發現當$(a, b, c)=(2, 1, 3)$時,$\lfloor\frac{100\max(a, b)}{\max(a, b)+c}\rfloor$有最小可能值$40$,所以\texttt{\$\{x\}}$=40$。
