\begin{pr}

\newcommand{\seq}[3]{\{{#1}_i\}_{i=#2}^{#3}}

First, let's solve the following problem:

Given $\seq a1n, \seq b1n, \seq c1n$, find $\sum_{(i, j)\text{ is an inversion in }\seq a1n}b_ic_j$ in $O(n\log n)$ complexity.

Let $d_{l, r}(b, c):=\sum_{(i, j)\text{ is an inversion in }\seq al{r-1}}b_ic_j$.

Let's implement $solve(l, r)$ such that it does the following things:
\begin{enumerate}
\item Sort $\seq al{r-1}, \seq bl{r-1}, \seq cl{r-1}$ by the order of $\seq al{r-1}$. (in other words, sort $\{(a_i, b_i, c_i)\}_{i=l}^{r-1}$ by $a_i$).
\item Return $d_{l, r}(b, c)$.
\end{enumerate}

Use divide and conquer to implement it.

For the base case $r\leq l+1$, just do nothing and return $d_{l, r}(b, c)=0$.

For the other case $r\geq l+2$, let $m:=\lfloor\frac{l+r}2\rfloor$.\\
First, do $solve(l, m)$ and $solve(m, r)$.\\
There are $3$ kinds of inversions $(i, j)$:
\begin{enumerate}[K-(1)]
\item $i<j<m$, the summation of $b_ic_j$ of this kind of inversions is exactly $d_{l, m}(b, c)$, which is counted by $solve(l, m)$.
\item $m\leq i<j$, the summation of $b_ic_j$ of this kind of inversions is exactly $d_{m, r}(b, c)$, which is counted by $solve(m, r)$.
\item $i<m\leq j$.
\end{enumerate}

Since $\seq al{m-1}, \seq am{r-1}$ have been sorted by $solve(l, m), solve(m, r)$, respectively, we can do the merge part in the merge sort to sort $\seq al{r-1}, \seq bl{r-1}, \seq cl{r-1}$ by $\seq al{r-1}$ in $O(r-l)$ time complexity.\\
Set $C$ to $0$ and $d_{l, r}(b, c)$ to $d_{l, m}(b, c)+d_{m, r}(b, c)$.\\
Do the following when merging $L:=\seq al{m-1}, R:=\seq am{r-1}$ to the sorted array $A$:
\begin{enumerate}[M-(1)]
\item If we put an element $a_i$ of $R$ to $A$, increase $C$ by $c_i$.
\item If we put an element $a_i$ of $L$ to $A$, increase $d_{l, r}(b, c)$ by $b_iC$.
\end{enumerate}
Note that for the tie breaker, we put the element in $L$ instead of that in $R$ to $A$, so that whenever an element $a_i$ of $L$ is put into $A$, $a_i>$ any element $a_j$ in $A$ that are from $R$, $a_i\leq$ any element $a_j$ that are not in $A$, and therefore $(i, j)$ forms an inversion of the third kind if and only if $a_j$ is in $A$ and is from $R$.\\
Since in M-(1) we maintain $C=\sum_{a_i\text{is from }R\text{ and is in }A}c_i$, we'll increase $d_{l, r}(b, c)$ by $\sum_{(i, j)\text{ is an inversion and }j\geq m}b_ic_j$ in M-(2).\\
$\so$ after merging $L, R$, the arrays are sorted, and we finish counting the summation of $b_ic_j$ of K-(3) $(i, j)$.\\
Return $d_{l, r}(b, c)$.

Since the time complexity for a single M-(1) or M-(2) is $O(1)$, and there are $O(r-l)$ elements to be merged, the time complexity of the merging part is $O(r-l)$.

Let $T(r-l)$ denote the time of $solve(l, r)$.\\
The time complexity of the dividing part is $2T((r-l)/2)$, of the merging part is $O(r-l)$.\\
$\then T(r-l)=2T((r-l)/2)+O(r-l)$.\\
By the master theorem, $T(r-l)=O((r-l)\log(r-l))$.\\
$\so T(n)=O(n\log n)$.

Back to (a), (b), (c):\\
(a) is $d_{l, r}(b, c)$, where $b_i:=c_i:=1$, which can be solved in $O(n\log n)$ time complexity.\\
Trivially, (b) can be solved if (c) is solved.\\
(c) is $\sum_{i=0}^k\binom kid_{l, r}(b^{(i)}, c^{(k-i)})$ by the binomial theorem, where $b_j^{(i)}:=c_j^{(i)}:=a_j^i$.\\
Since $\binom k0=1,\ \forall i,\ \binom k{i+1}=\binom ki\cdot\frac{k-i}{i+1}$.\\
$\so\binom k0, \binom k1, \dots, \binom kk$ can be counted in $O(k)$ time complexity.\\
Since each $d_{l, r}(b^{(i)}, c^{(k-i)})$ can be counted in $O(n\log n)$ time complexity, the total time complexity of (c) is $O(nk\log n+k)=O(nk\log n)$.
\end{pr}
