\begin{pr}$ $\\
(d)

Let $m:=\lfloor\frac n3\rfloor$.\\
Construction:\\
For $1\leq i\leq n-m$, the $i$-th set operation is to insert $i$.\\
For $n-m+1\leq i\leq n$, the $i$-th set operation is to delete $n-i+1$.

The number of stack operations:\\
In the first $n-m$ set operations, each contains one push operaion.\\
In the last $m$ set operations, the $i$-th one is to delete $n-i+1$, and before it, all delete operations are to delete $m, m-1, \dots, n-i+2$. Since the position of $n-i+1$ is under those of $m, m-1, \dots, n-i+2$, when deleting $m, m-1, \dots, n-i+2$, the position of $n-i+1$ won't be changed in Arctan's implementation, which means it will be under the position of $m+1, m+2, \dots, n$. Therefore, to delete $n-i+1$, Arctan needs to pop $m+1, m+2, \dots, n$ first, then pop $n-i+1$, finally push $m+1, m+2, \dots, n$ back to the stack, which takes $2(n-m)+1$ stack operations in total.\\
$\therefore$ the number of stack operations in total is $(n-m)+m(2(n-m)+1)=n-m+2nm-2m^2+m=n+2\lceil\frac{2n}3\rceil\lfloor\frac n3\rfloor=\Theta(n^2)$.

(e)

Define $B_{l, r}$ as $\left(\bigcup_{i=l}^{r-1}A_i\setminus A_{i+1}\right)\setminus\left(\bigcup_{i=l}^{r-1}A_{i+1}\setminus A_i\right)$. That is, the set of all elements that will be deleted but not be inserted during the $l$-th to the $r-1$-th set operation.\\
Define $C_{l, r}$ as $\left(\bigcup_{i=l}^{r-1}A_{i+1}\setminus A_i\right)\setminus\left(\bigcup_{i=l}^{r-1}A_i\setminus A_{i+1}\right)$. That is, the set of all elements that will be inserted but not be deleted during the $l$-th to the $r-1$-th set operation.\\
Define $S_{i, j}$ as $\begin{cases}
\text{the }j\text{-th element counted from the bottom of the stack }S_i\text{, if }j>0;\\
\text{the }(-j)\text{-th element counted from the top of the stack }S_i\text{, if }j<0.
\end{cases}$.\\
Define $S_{i, l..r}$ as the stack containing $r-l+1$ elements, where $\forall1\leq j\leq r-l+1$, the $j$-th element counting from the bottom is $S_{i, l+j-1}$.\\
Define $A_S$ as the set of the elements of a stack $S$.\\
Define $S+a$ as the stack formed by pushing an element $a$ into a stack $S$.

Let's implement $solve(l, r)$ such that it does the following things:
\begin{enumerate}[C-(1)]
\item Given is a stack $S_L$ such that $A_{S_L}=A_l$ and $A_{S_{L, -|B_{l, r}|..-1}}=B_{l, r}$.
\item Return is a stack $S_R$ such that $A_{S_R}=A_r$ and $S_{R, 1..|S_L|-|B_{l, r}|}=S_{L, 1..|S_L|-|B_{l, r}|}$.
\item $\exists i_l, i_{l+1}, \dots, i_r$ where $L=i_l<i_{l+1}<\cdots<i_r=R$ such that $\forall l\leq j\leq r,\ A_{S_{i_j}}=A_j$.
\end{enumerate}

Use divide and conquer to implement it.

For the base case $r\leq l+1$, if $A_{l+1}=A_l\cup\{a\}$ for some $a$, then let $S_R=S_{L+1}=S_L+a$; otherwise let $S_R=S_{L+1}=S_{L, 1..|S_L|-1}$ since by C-(1), $\{S_{L, -1}\}=A_l\setminus A_{l+1}$. One can easily check that C-(2) and C-(3) are satisfied.

For the other case $r\geq l+2$, let $k:=\lfloor\frac{l+r}2\rfloor$, and do the following:
\begin{enumerate}[O-(1)]
\item Pop the top $|B_{l, r}|$ elements from the stack, and the resulting stack is $S_{L+|B_{l, r}|}$.
\item Push all the elements in $B_{l, r}\setminus B_{l, k}$ into the stack, then push all the elements in $B_{l, k}$ into the stack, and the resulting stack is $S_{L_1}$, where $L_1:=L+2|B_{l, r}|$.
\item Do $solve(l, k)$. Since O-(1), O-(2) make $A_{S_{L_1}}=A_{S_L}=S_l$, and O-(2) gaurantees that $A_{S_{L_1, -|B_{l, k}|..-1}}=B_{l, k}$, C-(1) is satisfied. Suppose the returning stack is $S_{R_1}$.
\item Let $D:=B_{l, r}\setminus B_{l, k}\cup C_{l, k}$. Pop the top $|D|$ elements from the stack, and the resulting stack is $S_{R_1+|D|}$.
\item Push all the elements in $D\setminus B_{k, r}$ into the stack, then push all the elements in $B_{k, r}$ into the stack, and the resulting stack is $S_{L_2}$, where $L_2:=R_1+2|D|$.
\item Do $solve(k, r)$. Since O-(4), O-(5) make $A_{S_{L_2}}=A_{S_{R_1}}\overtext{C-(2)}=A_k$, and O-(5) gaurantees that $A_{S_{L_2, -|B_{k, r}|..-1}}=B_{k, r}$, C-(1) is satisfied. Suppose the returning stack is $S_{R_2}$.
\item Let $R:=R_2$, return $S_R$. Since by C-(2), $A_{S_R}=A_{S_{R_2}}=A_r$, and none of the above changes the $|S_L|-|B_{l, r}|$ elements in the bottom, C-(2) is satisfied. Since by C-(3), $solve(l, k), solve(k, r)$ gaurantee the existense of $i_l, i_{l+1}, \dots, i_r$, C-(3) is satisfied.
\end{enumerate}

Let $T(r-l)$ denote the number of stack operations in $solve(l, r)$.\\
For O-(1), (2), (4), (5), there are $2|B_{l, r}|+2|B_{l, r}|-2|B_{l, k}|+2|C_{l, k}|$ of stack operations in total. Since $B_{l, r}, B_{l, k}, C_{l, k}\leq r-l$, $2|B_{l, r}|+2|B_{l, r}|-2|B_{l, k}|+2|C_{l, k}|=O(r-l)$.\\
The number of stack operations in (3), (6) is $T((r-l)/2)$.\\
$\Rightarrow T(r-l)=2T((r-l)/2)+O(r-l)$.\\
By the master theorem, $T(r-l)=O((r-l)\log(r-l))$.\\
$\therefore m=T(n)=O(n\log n)$.
\end{pr}
