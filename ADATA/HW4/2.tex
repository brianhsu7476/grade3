\section*{(a)}
\noindent
Claim: All clauses can be satisfied $\iff$ for all variables $x$, either $x$ can't reach $\sim x$ or $\sim x$ can't reach $x$.\\
Proof:\\
($\Rightarrow$):\\
Since $a\lor b$ is true $\iff\sim a\to b$ is true, and if $a\to b$ and $b\to c$, then $a\to c$.\\
$\therefore a$ can reach $b$ on the graph means that $a\to b$ is true.\\
If $x$ can reach $\sim x$, then $\sim x\lor\sim x$ (which is $x\to\sim x$) is true.\\
$\Rightarrow x=0$.\\
Similarly, if $\sim x$ can reach $x$, then $x=1$.\\
$\therefore$ either $x$ can't reach $\sim x$ or $\sim x$ can't reach $x$.\\
($\Leftarrow$):\\
Consider the DAG of SCC.\\
Repeat choosing SCC with no out degree and assign all vertices in the SCC with true, and then remove that SCC from the DAG until we can't do that ($a$ is assigned to be true but $\sim a$ is in the SCC). Then we assign all left SCC with false.\\
Proof of $a, \sim a$ won't be false simultanously: if $a$ can't be true, it means that there is $b$ where $a$ can walk to $b$ such that $b$ is false and it's because of $\sim b$ is assigned true. Since this graph is symmetric ($a\to b\iff\sim b\to\sim a$), $\sim b$ can walk to $\sim a$ and therefore $\sim a$ has been assigned true.

It is equivalent to check if for each $a$, $\sim a$ is in the same SCC or not by the claim, which can be done by using Kosaraju's algorithm to find all SCC (with time complexity $O(n\log n)$) and then check the SCC that $a$, $\sim a$ are in for each $a$ (with time complexity $O(n)$).\\
This is an algorithm that runs in polynomial time, and therefore $2$-SAT is in P.

\section*{(b)}

\subsection*{(b-1)}

Let $\phi$ be any $2$-CNF formula, and $n(\phi):=$ the number of variables in $\phi$.\\
Let $\phi'$ be a $2$-CNF with $n(\phi)^{1000}$ variables, and $\phi'$'s clauses $:=$ the union of the clauses in $\phi$ and $\{x\lor\sim x:$ for all variables $x$ of $\phi'\}$.\\
One can see that $\phi'\in S_{x^{0.001}}$ by numbering the first $n(\phi)$ variables being those in $\phi$.\\
Since $x\lor\sim x$ is always true, the number of satisfied clauses in $\phi=$ the number of satisfied clauses in $\phi'-n(\phi)^{1000}$.\\
$\therefore(\phi, k)\in$ MAX-$2$-SAT $\iff(\phi', k+n(\phi)^{1000})\in$ MAX-$f$-CondSAT.\\
Also, $\phi'$ can be constructed from $\phi$ in complexity $n(\phi)^{1000}$, which is polynomial complexity.\\
$\therefore(\phi, k)\to(\phi', k+n(\phi)^{1000})$ is a polynomial reduction. Since MAX-$2$-SAT is NP-complete, MAX-$f$-CondSAT is also NP-complete.

\subsection*{(b-2)}

Sort all clauses by the maximum index of its variables.\\
Let $k:=\lfloor f(n)\rfloor$, and $j=(j_kj_{k-1}\cdots j_0)_2$.\\
Let $dp[i][j]:=$ the maximum number of clauses with both its variables in $\{a_1, a_2, \dots, a_i\}$ are satisfied given the condition $a_{i-l}=j_l,\ \forall0\leq l\leq k$.\\
$dp[i][j]=\max(dp[i-1][j/2], dp[i-1][j/2+1])+$ (the number of clauses containing $a_i, a_{i-l}$ are satisfied when $a_i=j_0$ and $a_{i-l}=j_l$ for all $l$).\\
After calculating the values of all $dp[i][j]$ in the order $i=1$ to $n$, $j=0$ to $2^{k+1}-1$, find the maximum value of $dp[n][j]$ among all $j$.\\
The time complexity of calculating a single $dp[i][j]$ is $O(k)$.\\
Suppose that $f(n)<c\log n$ for all $n$.\\
$\therefore$ the total time complexity $=O(n\times2^k\times k)=O(n\times2^{c\log n}c\log n)=O(n^{c+1}\log n)$, which is a polynomial time complexity.

\section*{(c)}

\subsection*{(c-1)}

Suppose that the variables used in $\phi$ are $a_1, a_2, \dots, a_n$.\\
Let $\phi'$ contains boolen variables $a_{i, j}$ for $i\in[n]$ and $j\in[c]$.\\
For all $i$ and $j_1\neq j_2$, add the clause $\sim a_{i, j_1}\lor\sim a_{i, j_2}$ to $\phi'$. These clauses are called type 1.\\
For each clause $a_i=b_1\lor a_j=b_2$ of $\phi$, add the clause $a_{i, b_1}\lor a_{j, b_2}$ to $\phi'$. These clauses are called type 2.\\
For an assignment $a_1, a_2, \dots, a_n$ such that $\phi$ is satisfied, set $a_{i, j}=\I\{a_i=j\}$. One can see that the clauses in type 2 are satisfied, and the clauses in type 1 are satisfied because $a_i=j_1, a_i=j_2$ won't be true simultanously for $j_1\neq j_2$.\\
For an assignment of all $a_{i, j}$ such that $\phi'$ is satisfied, set $a_i=\begin{cases}
j\text{, if }\exists j\text{ s.t. }a_{i, j}\text{ is true}\\
1\text{, otherwise}
\end{cases}$.\\
Such existence of $j$ is unique because $\sim a_{i, j_1}\lor\sim a_{i, j_2}$ for all $j_1\neq j_2$. If $a_{i, j}$ is false for all $j$, then assigning $a_{i, 1}=1$ won't change any clause of $\phi'$ to false.\\
Since $a_{i, b_1}\lor a_{j, b_2}$ is true implies $a_i=b_1\lor a_j\lor b_2$ is true, all clauses in $\phi$ are satisfied.\\
$\therefore\phi\in c-$GenSAT$\iff\phi'\in2$-SAT.\\
Also, constructing $\phi'$ from $\phi$ takes polynomial time complexity.\\
$\therefore\phi\to\phi'$ is a polynomial reduction. Since $2$-SAT is in P, $c$-GenSAT is also in P.

\subsection*{(c-2)}

Let's have a reduction from $3$-colorability problem to $c$-GenSetSAT.\\
For a graph $G$ with vertices $v_1, v_2, \dots, v_n$, construct a correspond formula $\phi$ with variables $a_1, a_2, \dots, a_n$. If $v_iv_j\in G$, add the three clauses $a_i\in\{1\}\lor a_j\in\{2, 3\}, a_i\in\{2\}\lor a_j\in\{3, 1\}, a_i\in\{3\}\lor a_j\in\{1, 2\}$ to $\phi$.\\
One can see that the above three clauses are all satisfied iff $a_i\neq a_j$ and $a_i, a_j\in\{1, 2, 3\}$.\\
Therefore, $\phi$ is true iff $a_i\neq a_j$ for all edges $v_iv_j$ iff we can color $v_i$ in $a_i$ for some $a_i$.\\
$\therefore G$ can be $3$-colored iff $\phi\in3$-GenSetSAT.\\
Also, constructing $\phi$ from $G$ takes polynomial time complexity.\\
$\therefore G\to\phi$ is a polynomial reduction. Since $c$-colorability problem is NP-complete, $3$-GenSetSAT is also NP-complete.
