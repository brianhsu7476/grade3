\begin{pr}
Let $V$ denote the vertex set, and $E$ denote the edge set.\\
Algorithm:\\
For every vertex, color it with one of the $k$ colors uniform randomly and independently.\\
For every edge $(u, v)$, $\P[(u, v)\in S]=\P[u, v$ have the different colors $]=1-\frac1k$.\\
$\so$ the expected size of $S$ is $(1-\frac1k)|E|\geq(1-\frac1k)OPT$, and this is a randomized $(1-\frac1k)$-approximation algorithm.\\
Derandomize:\\
Suppose that $V=[n]$.\\
Let $[k]$ denote the $k$ colors.\\
Run the following algorithm with parameter $m$ to obtain the coloring $c_m:[n]\to[k]$.\\
When $m=n$, the algorithm is deterministic.
\begin{itemize}
\item for $i=1$ to $m$:
\begin{itemize}
\item Choose $j$ s.t. $|\{1\leq i'\leq i-1:c_m(i')\neq j, (i', i)\in E\}|$ is maximized. -- (1)
\item Set $c_m(i)=j$ (i.e. color the vertex $i$ with $j$).
\end{itemize}
\item for $i=m+1$ to $n$:
\begin{itemize}
\item Uniformly randomly choose $j$ from $k$.
\item Set $c_m(i)=j$ (i.e. color the vertex $i$ with $j$).
\end{itemize}
\end{itemize}
Let the resulting $S$ of the algorithm be $S_m$.\\
$\E[|S_0|]\geq(1-\frac1k)OPT$ has been proved above.\\
$\E[|S_i|]=|\{(u, v)\in E:u<v<i, c_i(u)\neq c_i(v)\}|+|\{(u, i)\in E:u<i, c_i(u)\neq c_i(i)\}|+(1-\frac1k)|\{(u, v)\in E:u<v, v>i\}|\\
\overtext{By (1)}\geq|\{(u, v)\in E:u<v<i, c_i(u)\neq c_i(v)\}|+(1-\frac1k)|\{(u, i)\in E:u<i\}|+(1-\frac1k)|\{(u, v)\in E:u<v, v>i\}|=\E[|S_{i-1}|]$.\\
$\so$ the algorithm with parameter $m=n$, which is a deterministic algorithm, satisfied $|S_n|=\E[|S_n|]\geq\E[|S_{n-1}|]\geq\cdots\geq\E[|S_0|]\geq(1-\frac1k)OPT$.\\
Clearly, setting $c_n(i)=j$ above is $O(1)$.\\
One can first store the neighborhood of each vertex, and then in (1), run through all neighbors of $i$.\\
Since each edge will be run for twice in (1), the running time of this algorithm is $O(|V|+|E|)$.
\end{pr}
