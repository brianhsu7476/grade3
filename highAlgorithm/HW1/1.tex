\begin{pr}
Let OPT be the optimal solution, and $val(OPT)$ be its value.\\
If $c_1+(n-1)c_m>B$, then no worker can select the first item, and we can remove all $p_{11}, p_{21}, \dots, p_{n1}$, so let's suppose that $c_1+(n-1)c_m\leq B$.\\
Let $b=\frac{p_{11}\epsilon}{2n}$, and $p_{ij}':=\lceil\frac{p_{ij}}b\rceil b$, which we'll call it "new productivity".\\
Let $q_{ij}:=\frac{p_{ij}'}b=\lceil\frac{p_{ij}}b\rceil$.\\
Let $dp_{ij}:=$ the minimum cost that can achieved with new productivity $jb$ by $W_1, W_2, \dots, W_i$, and $r_{ij}$ the item that $W_i$ should select to achieve such minimum cost. The range: $1\leq i\leq n,\ 0\leq j\leq Q$, $Q:=\suml_{k=1}^nq_{k1}$.\\
The base case $i=1$: $$dp_{1j}=\begin{cases}
\min_{k:q_{1k}=j}(c_k)\text{, if }\exists k\text{ s.t. }q_{1k}=j\\
B+1\text{, otherwise}
\end{cases}$$
$$r_{1j}=\begin{cases}
k\text{, where }c_k=dp_{1j}\text{ and }q_{1k}=j\text{, if such }k\text{ exists}\\
-1\text{, otherwise}
\end{cases}$$
One can run from $i=2$ to $n$, from $j=0$ to $Q$ to get the values of $dp_{ij}$ using
$$dp_{ij}=\begin{cases}
\min_{k:q_{ik}\leq j}(dp_{i-1, j-q_{ik}}+c_k)\text{, if }\exists k\text{ s.t. }q_{ik}\leq j\\
B+1\text{, otherwise}
\end{cases}$$
$$r_{ij}=\begin{cases}
k\text{, where }dp_{i-1, j-q_{ik}}+c_k=dp_{ij}\text{ and }q_{ik}\leq j\text{, if such }k\text{ exists}\\
-1\text{, otherwise}
\end{cases}$$
Denote the optimal solution as ALG, and the value of ALG (denote as $val'(ALG)$ is the maximum new productivity that can be achieved with cost at most $B$, which is $\max_{j:dp_{nj}\leq B}(jb)$, and we can recursively find the selected item that can achieve this using $r_{ij}$.\\
The above can be done in $O(nQm)$ time complexity.

Let $j_i$ denote the selected item by $W_i$ in ALG, and let $\suml_{i=1}^np_{ij_i}$ be the productivity value of this algorithm (denoted as $val(ALG)$).\\
Let $k_i$ denote the selected item by $W_i$ in OPT, and let the new productivity value of these selected item be $val'(OPT)$.\\
By the definition of OPT, $val(ALG)\leq val(OPT)$.\\
Since the above dp algorithm obtains optimal solution of new productivity, $val'(ALG)\geq val'(OPT)$.\\
$val(ALG)=\suml_{i=1}^np_{ij_i}>\suml_{i=1}^n(\lceil\frac{p_{ij_i}}b\rceil-1)b=val'(ALG)-nb\geq val'(OPT)-nb=\suml_{i=1}^n\lceil\frac{p_{ik_i}}b\rceil b-nb\geq\suml_{i=1}^np_{ik_i}-nb=val(OPT)-nb=val(OPT)-\frac{p_{11}\epsilon}2$.\\
By what we suppose in the first three lines, $p_{11}\leq p_{11}+p_{2m}+p_{3m}+\cdots+p_{nm}\leq val(OPT)$ (since $W_1$ can select $1$, while $W_2, W_3, \dots, W_n$ select $m$).\\
$\then val(ALG)\geq val(OPT)-\frac{val(OPT)\epsilon}2=(1-\frac{\epsilon}2)val(OPT)$.\\
$\then val(ALG)\leq val(OPT)\leq\frac{val(ALG)}{1-\frac{\epsilon}2}\leq(1+\epsilon)val(ALG)$.\\
The time complexity of this algorithm $=O(nQm)=O(n\suml_{k=1}^nq_{k1}m)=O(n\suml_{k=1}^n\lceil\frac{p_{k1}}b\rceil m)=O(n\suml_{k=1}^n\lceil\frac{2np_{k1}}{p_{11}\epsilon}\rceil m)=O(n\suml_{k=1}^n\frac{2n}\epsilon m)=O(\frac{n^3m}\epsilon)$
\end{pr}
